%Chapter 1

\renewcommand{\thechapter}{1}

\chapter{The State of the Art}

\section{Resilience Definitions}

\subsection{Mechanical Engineering}
Mechanical Engineering has a very specific definition of resilience as
a characteristic of a material. Resilience is the ability of a
material ``to store or absorb energy without permanent deformation''
\cite{Popov1964}.
The modulus of resilience is a property used to compare
materials for use in applications where energy must be absorbed by the
component without permanently altering the component's geometry. When
the plastic deformation is taken into account, the area under the
stress strain curve until fracture is a material's toughness. Both of
these concepts have use in our exploration. As we will see, the
modulus of resilience is analogous to engineering resilience
\cite{Holling1973a}
while toughness is analogous to ecological resilience.

\subsection{Ecology}
The field of ecology has a rich history with resilience, and many
concepts and applications may be portable to the engineering domain,
especially when considering complex systems' resilience such as
infrastructure resilience, community resilience, and System of Systems
(SoS) resilience. Multiple papers have laid out the history of this
thinking, and the resilience mindset is a common discussion item by
practitioners and leading researchers in the field \cite{Curtin2014}.
explore the history of resilience-thinking in ecology. The
narrative progresses from introductory concepts of resilience through
more complex iterations including panarchy, and finally prospects for
action in adaptive management. The problem of ecological management is
treated through Holling first introduced resilience to ecology in his
seminal work Resilience and Stability of Ecological Systems (Holling,
1973). 
