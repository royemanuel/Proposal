%Chapter 1

\renewcommand{\thechapter}{1}

\chapter{The State of the Art}

\section{Resilience Definitions}

\subsection{Mechanical Engineering}
Mechanical Engineering has a very specific definition of resilience as
a characteristic of a material. Resilience is the ability of a
material ``to store or absorb energy without permanent deformation''
\cite{Popov1964}.
The modulus of resilience is a property used to compare
materials for use in applications where energy must be absorbed by the
component without permanently altering the component's geometry. When
the plastic deformation is taken into account, the area under the
stress strain curve until fracture is a material's toughness. Both of
these concepts have use in our exploration. As we will see, the
modulus of resilience is analogous to engineering resilience
\cite{Holling1973a}
while toughness is analogous to ecological resilience.

\subsection{Ecology}
The field of ecology has a rich history with resilience, and many
concepts and applications may be portable to the engineering domain,
especially when considering complex systems' resilience such as
infrastructure resilience, community resilience, and System of Systems
(SoS) resilience. Multiple papers have laid out the history of this
thinking, and the resilience mindset is a common discussion item by
practitioners and leading researchers in the field \cite{Curtin2014}.
explore the history of resilience-thinking in ecology. The
narrative progresses from introductory concepts of resilience through
more complex iterations including panarchy, and finally prospects for
action in adaptive management. The problem of ecological management is
treated through Holling first introduced resilience to ecology in his
seminal work Resilience and Stability of Ecological Systems (Holling,
1973).

\cite{Holling1973a} proposed two types of resilience: engineering
resilience and ecological resilience.

\subsubsection{Engineering Resilience}
Engineering resilience is a system's ability to return to status quo
performance \cite{Holling2010}. Engineering resilience is similar to
the mechanical engineering concept of modulus of resilience because
both concepts describe behavior against a status quo prior to the
changing force.

\subsubsection{Ecological Resilience}
Ecological resilience is a system's ability to persist through input
changes and to establish different equilibria after a disturbance. The
equilibria may be called basins of attraction in the ecological
literature. The terminology, basins of attraction, better describes
the system behavior of socio-ecological systems. The relationship
between entities in the system, such as predator and prey populations,
are not steady equilibria. The values cycle about the domain of
attraction. A system may have multiple basins of attraction. When a
disturbance changes the inputs to the system, the system undergoes
regime change, and the trajectory of the system in its state space
transitions around a new basin of attraction \cite{Folke2010a}. While
more complex, this definition of resilience shares similarities to the
definition of toughness because both concepts describe system
persistence through disturbance while accumulating changes to its
composition.

\subsection{Risk Analysis}
The Society of Risk Analysis has defined resilience with a family of
definitions \cite{Aven2015b}:
\begin{quotation}
  \begin{itemize}
    \item Resilience is the ability of the system to sustain or
      restore its basic functionality following a risk source or an
      event (event unknown).
    \item Resilience is the sustainment of the system's operations
      and associated uncertainties, following a risk source or an
      event (event unknown).
    \item Resilience is the ability of a system to reduce the initial
      adverse effects (absorptive capability) of a disruptive event
      (stressor) and the time/speed and costs at which it is able to
      return to an appropriate functionality/equilibrium (adaptive and
      restorative capability).
  \end{itemize}
\end{quotation}
