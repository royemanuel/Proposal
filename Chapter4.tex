%Chapter 4

\renewcommand{\thechapter}{4}

\chapter{Methodology}

\section{Literature Review}
I want to build a new metric incorporating the things above and apply
it to situations.
\begin{enumerate}
\item Fleets of systems(availability and resilience)
  \item Community resilience (life-line resilience and infrastructure
    resilience)
  \item Cyber???
\end{enumerate}

\subsection{Basic Metric}
Ayyub \cite{Ayyub2014a} and Ouyang \emph{et al.} \cite{Ouyang2012}
present similar metrics for resilience both based on the ratio of the
integrals of actual performance over time and desired performance over
time. While the metric in Ouyang \emph{et al.} is elegant in its
simplicity, the metric from Ayyub (2014a) parses the phases of system
response into segments before, during, and after a disturbance. The
investigation will continue by modifying the Ayyub metric \textbf{may
  want to consider this more of a pointer to a particular equation
  rather than continuously referencing Ayyub}.

\subsection{Proposed metric}
The following metric based on Ayyub \cite{Ayyub2014a, Ayyub2015} and
Ouyang \emph{et al.} \cite{Ouyang2012} is proposed:
$$R = \frac{\int_{t_0}^{t_h}P_T(t)dt}{\int_{t_0}^{t_h}P_D(t)dt}$$
where
\[P_T(t)=
\begin{cases}
  P_R(t) & \text{if } P_R(t) \le P_D(t)\\
  P_D(t) & \text{if } P_R(t) > P_D(t)
\end{cases}\]




\section{Metric Requirements}
