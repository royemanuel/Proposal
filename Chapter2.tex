%Chapter 2

\renewcommand{\thechapter}{2}

\chapter{Resilience Definition and Components}

\section{Roots in Safety, Reliability, and Risk}

\section{Resilience}
First we define resilience. Multiple domains use resilience including
psychology, metallurgy, ecology, economics, and
engineering. Engineering is making great use of the concept including
control theory, civil engineering, and network engineering. Community
resilience is a current cross-discipline activity. Because there are
many different sources making demands upon this concept, we will
explore the use of resilience using stasis theory.

\subsection{Does Resilience Exist?}

Many communities make use of the concept of resilience, so its
existence is not in doubt. Resilience is some characteristic of an
entity, referred to as entity-of-interest. 

\subsection{Definition}

A commonality among all fields, even ones as different as
psychology and metallurgy, is the entity-of-interest's ability to
return to a desirable state after being forced into an undesirable
state by some disturbance. This is usually some level of performance
or measure of wellness for the entity-of-interest that is challenged
by a disturbance.

Figure~\ref{res} is an example of Pop-Art.
\begin{figure}
  \centering
  \includegraphics[natwidth=0.8\textwidth]{res.png}
  \caption{Five by Five in Centimetres.\label{res}}
\end{figure}
Each community has its own particulars or details attached to
resilience.Three major components of reliability vary among the domains. 

\begin{enumerate}
\item Resistance and absorption - Some conceptualizations of
  resilience include the entity-of-interest's ability to absorb or
  resist a disturbance so it does not experience the undesirable state
  in the first place. Other conceptualizations of resilience consider
  only the behavior of the entity after the disturbance and
  undesirable state occur.
  
\item Time Dependence - some of the engineering metrics associated
  with resilience do not include a time dependence for resilience. 
  
\item Recovery - some engineering metrics, particularly the ones
  associated with Systems of Systems (SoS), only consider the failure
  conditions of the SoS as systems in the SoS fall offline. The
  profile for recovery is not considered.
  
  
\end{enumerate}

\subsection{Quality and Value of Resilience}

Most every field considers resilience to be a desired characterstic of
an entity. One particular divergence from this assumption is the
engineering resilience in the field of ecology. This is seen as a
simplistic point of view of resilience and inappropriate for the
ecological domain.

One issue that has not been considered in the literature is that of
stopping criteria. We live in a finite world, and every decision and
associated action has an opportunity cost. Little, if any, analysis
has been done on what limits to resilience exist. Most domains,
particularly those concerned with community resilience and other
complex systems begin with the assumption that the return of a
community to its previous heights after a disaster is desired. One
could imagine a countering scenario, albeit an extreme example, where
a nuclear accident has rendered an area uninhabitable so resilience is
not a desired characteristic for the community as much as it is for
the individuals of the community to find a satisfactory alternative to
their previous living arrangement.  This example shows that a need
exists for buildign stopping criteria into a resilience analysis.

The value of resilience, as in how it is measured, has sparked many
journals. Common discussion of resilience based groups spend a good
portion of their time trying to describe the what and the definition
and the valuation of resilience.

Different metrics for resilience have different requirements driving
their development and have different characteristics when formulated. 


\cite{Barker2013}
